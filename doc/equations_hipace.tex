\documentclass{article}
\usepackage[a4paper, total={6in, 8in}]{geometry}
\usepackage[utf8]{inputenc}

\title{Contemporary field equations in HiPACE}

\usepackage{xcolor}
\usepackage{natbib}
\usepackage{graphicx}
\usepackage{amsmath}
\usepackage{bm}
\begin{document}


\section*{Contemporary field solving routine in HiPACE}
\label{se:pic_rad_field_solver}

\newcommand{\blue}[1]{\textcolor{blue}{#1}}
\newcommand{\red}[1]{\textcolor{red}{#1}}

\blue{In what follows, blue symbols are used for SI symbols. Discard them to have the normalized equations.}\red{Red factors are required for normalized units only. Discard them to have SI equations.}

\subsection*{Field equation}

In the beginning of the plasma routine, the plasma particles are pushed and the current and charge densities are deposited. All fields are calculated from there.
The transverse focusing field will be calculated from $-\Psi$, which can be calculated from the plasma charge density $\rho_p$ and the longitudinal current density $J_z$:
\begin{equation}
\nabla^2_{\perp} (- \Psi)  = \blue{\frac{1}{\epsilon_0}}\rho_p - \blue{\frac{1}{c}}J_z .
\label{eq:psi}
\end{equation}
Note: the beam contribution is not needed in this equation. In the ultra-relativistic limit, neglecting the transverse currents within the beam, the charge density of the beam $\rho_b$ equals $J_{z,b}$ and cancels out. The contribution of the beam to $J_z$ is added after $-\Psi$ is calculated.
In 3D, equation \ref{eq:psi} is solved with Poisson solvers, in RZ it is solved with a tridiagonal matrix solver.
The focusing field is then calculated by
\begin{subequations}
\begin{align}
    E_x - \blue{c}B_y &= \frac{\partial}{\partial x} (-\Psi) \\
    E_y + \blue{c}B_x &= \frac{\partial}{\partial y} (-\Psi)
\end{align}
\end{subequations}


The longitudinal accelerating field $E_z$ will be calculated from
\begin{equation} \label{eq:ez}
\nabla^2_{\perp} E_z = \blue{\frac{1}{\epsilon_0 c}} \nabla_{\perp} \cdot  J_{\perp} .
\end{equation}
In full 3D, this is realized by
\begin{equation}
    \nabla_{\perp} \cdot J_{\perp} = \frac{\partial J_x}{\partial x} +  \frac{\partial J_y}{\partial y}
\end{equation}
In the radial solver, it is calculated in cylindrical coordinates:
\begin{equation}
    \nabla_{\perp} \cdot J_{\perp} = \frac{\partial J_r}{\partial r} + \frac{J_r}{r}.
\end{equation}
Equation \ref{eq:ez} is then again solved via Poisson solvers in 3D and with a tridiagonal matrix solver in RZ.

The magnetic fields are calculated with:
\begin{equation} \label{eq:b_trans}
\nabla_{\perp}^2 B = - \blue{\mu_0} \nabla \times J .
\end{equation}

In full 3D this can be written as
\begin{subequations}
\begin{align}
\nabla_{\perp}^2 B_x  &= \blue{\mu_0}(- \partial_y J_z + \partial_{\zeta} J_y) \,  \\
\nabla_{\perp}^2  B_y  &=   \blue{\mu_0}(\partial_x J_z - \partial_{\zeta} J_x)   \, \\
 \nabla^2_{\perp} B_z &= \blue{\mu_0}(\partial_y J_x - \partial_x J_y).
\end{align}
\end{subequations}
The longitudinal magnetic field $B_z$ is only non-trivial in a full 3D simulation and vanishes in RZ. Whereas the longitudinal field $B_z$ can be calculated directly, the transverse $B$-fields require a predictor-corrector loop, as described below.
In cylindrical coordinates, equation \ref{eq:b_trans} transforms to
\begin{equation}
\frac{\partial^2 B_{\phi}}{\partial r^2} + \frac{1}{2} \frac{\partial B_{\phi}}{\partial r} - \frac{B_{\phi}}{r^2} = \frac{\partial J_z}{\partial r} - \frac{\partial J_r}{\partial \zeta}.
\end{equation}

Since the longitudinal derivative $\partial / \partial \zeta$ is not directly obtainable, a initial guess for the transverse $B$-fields is generated from the $B$-field from the previous slice and the $B$-field from the current slice from the previous time step (if there are earlier time steps). Then, the particles are pushed to the next slice with the guessed $B$-field and the currents are deposited. Having the current at the next slice, the derivative $\partial / \partial \zeta$ can be calculated, enabling the calculation of the $B$-field. The $B$-fields are re-calculated this way until a certain convergence criteria for the updated $B$-field and the previous iteration is matched.

\subsection*{Charge (and current) deposition}

In general, the charge deposition is the operation
\begin{equation}
\rho_{ijk} += \frac{q N S(x_{ijk})}{\Delta V_{cell}}
\end{equation}
where $q$ is the charge of 1 physical particle of the species, $N$ is the number of physical particles in current macro-particle, $S(x)$ is the value of the shape factor at the grid point $ijk$ and $\Delta V$ is the volume of 1 cell.

In Hipace, this is interpreted as
\begin{equation}
\begin{split}
\rho_{ijk} &+= \frac{q N S(x_{ijk})}{\Delta V_{cell}} \\
           &+= \frac{q N}{\Delta V_{cell}} S(x_{ijk}) \\
           &+= \rho_{particle} S(x_{ijk}) \\
           &+= wS(x_{ijk})
\end{split}
\end{equation}
where $w$ is the weight of the particle given by
\begin{equation}
\begin{split}
w &= \blue{\frac{q N}{\Delta V_{cell}}}\frac{1}{ppc} \\
\end{split}
\end{equation}
where $ppc$ is the number of particles per cell. In this case, a particle is interpreted as a small amount of the charge density.

In WarpX, this is interpreted as
\begin{equation}
\begin{split}
\rho_{ijk} &+= \frac{q w S(x_{ijk})}{\Delta V_{cell}} \\
w &= N \text{ in S.I.} \\
\end{split}
\end{equation}
Note that this interpretation is very handy in SI (because the weight has a straightforward definition) but may not be that well adapted for normalized units because $\rho$ is normalized by $n_0$ while the cell size, used for the volume, is normalized by $k_p$, so we end up with inelegant $n_0k_p^3$ (I think, from Maxence). We could probably have an if statement there, to change between normalized and S.I.

\subsection*{Plasma particle pushers}

Equations (3.9) to (3.12) in Timon Mehrling's manuscript read, respectively,
% \frac{e\psi_p}{m_ec^2}

\begin{equation}
\partial_{\zeta}X_p = -\blue{\frac{1}{Mc}}\frac{U_p} {1+\blue{\frac{e}{m_e c^2}}\psi_p }
\end{equation}

\begin{equation}
\partial_{\zeta}U_p =
\red {\frac{M}{m_e}}
\blue{\frac{e}{c}}
\left[
\frac{\gamma_p}{1+\blue{\frac{e}{m_ec^2}}\psi_p}
\left(
\begin{array}{l}
   E_x - \blue{c}B_y \\
   E_y + \blue{c}B_x
\end{array}
\right)
+
\left(
\begin{array}{l}
   +\blue{c}B_y \\
   -\blue{c}B_x
\end{array}
\right)
+
\blue{\frac{1}{Mc}}
\frac{\blue{c}B_z}{1+\blue{\frac{e}{m_ec^2}}\psi_p}
\left(
\begin{array}{l}
   +U_y \\
   -U_x
\end{array}
\right)
\right]
\end{equation}

\begin{equation}
\partial_{\zeta}\psi_p =
\blue{\frac{1}{Mc}}
\frac{U_p}{1+\blue{\frac{e}{m_ec^2}}\psi_p}
\left(
\begin{array}{l}
   E_x - \blue{c}B_y \\
   E_y + \blue{c}B_x
\end{array}
\right)
-
E_z
\end{equation}

\begin{equation}
\gamma_p = \frac{1 + \blue{\frac{1}{M^2c^2}}U_p^2 + \left( 1+\blue{\frac{e}{m_ec^2}} \psi_p \right)^2}
                {2\left(1+\blue{\frac{e}{m_ec^2}}\psi_p\right)}
\end{equation}

\pagebreak

\subsection*{Algorithm}

Subscript $p$ means plasma, subscript $b$ means beam. Brackets in superscript $(n==+2)$ mean "predicted". $\rho$ and capital letters are field data, others are particle data. $x$, $u$ and $\psi$ are the position, momentum and pseudo-potential respectively. $f$ stands for the force terms.

\begin{subequations}
\begin{align}
\text{plasma push: } f_x^n, f_u^n, f_{\psi}^n &\longrightarrow x_p^{n+1}, u_p^{n+1}, \psi_p^{n+1}, x_{prev}=x_p^{n+1} \\
\text{plasma deposition: } x_p^{n+1}, u_p^{n+1} &\longrightarrow \rho_p^{n+1}, J_{p,\perp}^{n+1}, J_{p,z}^{n+1} \\
\text{Poisson solve: } \rho_p^{n+1}, J_{p,z}^{n+1} &\longrightarrow \Psi_p^{n+1} \longrightarrow (E_x-cB_y)^{n+1}, (E_y+cB_x)^{n+1}\\
\text{beam deposition: } x_b^{n+1}, u_b^{n+1} &\longrightarrow J_{\perp}^{n+1}, J_z^{n+1} \\
\text{Poisson solve: } J_{\perp}^{n+1} &\longrightarrow E_z^{n+1}, B_z^{n+1} \\
\text{Init Pred-Corr guess: } B_{\perp}^{n-1}, B_{\perp}^{n} &\longrightarrow B_{\perp}^{(n+1)}, B_{\perp}^{prev}=B_{\perp}^{(n+1)} \\
\text{update plasma forces: } E^{n+1}, B_z^{n+1}, B_{\perp}^{(n+1) }&\longrightarrow f_x^{(n+1)}, f_u^{(n+1)}, f_{\psi}^{(n+1)}
\end{align}
\text{Predictor-corrector loop, while error$(B_{\perp, iter}^{(n+1)}, B_{\perp}^{(n+1)}) >$ tolerance:}
\begin{align}
\text{plasma push: } x_p^{n+1}, u_p^{n+1}, \psi_p^{n+1}, f_x^{(n+1)}, f_u^{(n+1)}, f_{\psi}^{(n+1)} &\longrightarrow x_p^{(n+2)}, u_p^{(n+2)}, \psi_p^{(n+2)} \\
\text{plasma deposition: }x_p^{(n+2)}, u_p^{(n+2)}, \psi_p^{(n+2)} &\longrightarrow J_{\perp}^{(n+2)} \\
\text{Poisson solve: }J_z^{n+1}, J_{\perp}^{n}, J_{\perp}^{(n+2)} &\longrightarrow B_{\perp, iter}^{(n+1)}\\
\text{New guess: }B_{\perp}^{n}, B_{\perp, iter}^{(n+1)}, B_{\perp}^{prev} &\longrightarrow B_{\perp}^{(n+1)}, B_{\perp}^{prev}=B_{\perp, iter}^{(n+1)} \\
\text{update plasma forces: }B_{\perp}^{(n+1)}, E_{\perp}^{n+1} &\longrightarrow f_x^{(n+1)}, f_u^{(n+1)}, f_{\psi}^{(n+1)}
\end{align}
\end{subequations}

\subsection*{WAND-PIC method}


The WAND-PIC paper (open-source implementation available, see below) proposes a new approach to get rid of all steps of the predictor-correct loop above, and derives an equation (albeit a bit more complicated than a Poisson equation) for $B_x$ and $B_y$. This is equation (19) of their paper:

\begin{subequations}
\begin{align}
\Delta_\perp \bm{B_\perp} &= \frac{n_*}{1+\psi} \bm{B_\perp} - \bm{e_z} \times \bm{S} \\
\bm{S}_\perp &= \frac{1}{1+\psi} B_z \left( \bm{e_z} \times n_* \langle\bm{v_\perp}\rangle \right) + n_* \langle \bm{\tilde{a}} \rangle - \partial_x\left( n_*\langle v_x\bm{v_\perp} \rangle \right) - \partial_y\left( n_*\langle v_y\bm{v_\perp} \rangle \right) + \bm{\nabla_\perp}j_z^{total} \\
n_* \langle \bm{\tilde{a}} \rangle &= \frac{1}{1+\psi} \left(  \frac{n_*\langle\gamma\rangle}{1+\psi}\bm{\nabla_\perp}\psi - n_* \langle \bm{v_\perp} \rangle E_z - n_* \langle v_x\bm{v_\perp}\rangle \partial_x\psi - n_* \langle v_y\bm{v_\perp}\rangle \partial_y\psi \right) \\
n_*\langle \gamma\rangle &= \frac{1+\psi}{2}\left(n_*\langle v_x^2\rangle + n_*\langle v_y^2\rangle + n_*\right) + \frac{n_*}{2(1+\psi)}
\end{align}
\end{subequations}

where $n_* = n_e - j_z$ where $n_e$ is the plasma electron density and $jz$ is the longitudinal current density. All densities are only plasma densities (charge and current), except when $total$ is specified in subscript, in that case it is the full density (plasma + beam). The notations used here are from the WAND-PIC paper.

The correspondance between quantities Hipace++ and WAND-PIC is:
\begin{subequations}
\begin{align}
\text{WAND-PIC} &\leftrightarrow \text{Hipace++} \\
n_* &\leftrightarrow \rho - j_z \text{ (+ ions)} \\
n_e &\leftrightarrow \rho \text{ (+ ions)}  \\
n_*\langle \bm{v_\perp}\rangle &\leftrightarrow \bm{J_\perp} (J_x \text{ and } J_y) \\
n_*\langle v_x \bm{v_\perp}\rangle \text{ and } n_*\langle v_y \bm{v_\perp}\rangle &\leftrightarrow \text{does not exist } (J_{xx}, J_{yy}, J_{xy})
\end{align}
\end{subequations}

The main equation to solve is the first one. It is an inhomogeneous Helmholtz equation with non-constant coefficients, unfortunately I can't see any way to solve it with FFT, so the plan is to use AMReX' Multigrid solver.

\begin{itemize}
\item WAND-PIC paper: https://arxiv.org/abs/2012.00881
\item WAND-PIC repo: https://github.com/tianhongg/WAND-PIC
\end{itemize}

\end{document}
